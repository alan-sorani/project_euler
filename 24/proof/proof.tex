\documentclass[10pt]{article}

%---HYPERLINKS---

\RequirePackage{hyperref}
\hypersetup{
    colorlinks=true, %set true if you want colored links
    linktoc=all     %set to all if you want both sections and subsections linked
}

%---MATH---

\usepackage{math}

%---BIBLIOGRAPHY---

\usepackage[backend=biber, style=alphabetic]{biblatex}
\addbibresource{bibliography.bib}

%---TITLE---

\author{Alan Sorani}
\title{Proof for Counting Lexicographic Permutations}

\begin{document}
\maketitle

We start by stating a well-known theorem.

\begin{theorem}
The number of permutations of a set of $n$ elements are $n!$.
\end{theorem}

\begin{corollary}
The number of permutations of a set of $n$ elements such that the first element is fixed is $\prs{n-1}!$.
\end{corollary}

\begin{proof}
Permutations of $n$ elements such that the first element is fixed, correspond directly to permutations of the other elements.
\end{proof}

\begin{corollary}
In order to find the $k$\textsuperscript{th} permutation of the set $\set{0, 1, \ldots, 9}$, in lexicographic order, we may follow the following algorithm.

\begin{enumerate}
\item Let $m$ be the number of elements in the set $\set{0, \ldots, 9}$ minus one.
\item Set the left-most remaining digit to $\ceil{\frac{k}{m!}}$\textsuperscript{th} smallest digit.
\item Change the value of $k$ to $k - \prs{\ceil{\frac{k}{m!}} - 1} \cdot m!$, and decrease the value of $m$ by $1$.
\item If there are more digits to determine, go back to step $2$ with the rest of the digits available.
\end{enumerate}
\end{corollary}

\begin{proof}
Each digit appears as the left-most digit in the first $m!$ permutations, where $m$ is the number of digits left to look at.
The left-most digit is therefore determined by the number of times $9!$ goes into $k$.

The next digit is determined similarly by the number of times $8!$ goes into how many permutations are left to count from the first one with the first digit being fixed, etc.
\end{proof}



\printbibliography

\end{document}