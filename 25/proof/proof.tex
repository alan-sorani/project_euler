\documentclass[10pt]{article}

%---HYPERLINKS---

\RequirePackage{hyperref}
\hypersetup{
    colorlinks=true, %set true if you want colored links
    linktoc=all     %set to all if you want both sections and subsections linked
}

%---MATH---

\usepackage{math}

%---BIBLIOGRAPHY---

\usepackage[backend=biber, style=alphabetic]{biblatex}
\addbibresource{bibliography.bib}

%---TITLE---

\author{Alan Sorani}
\title{Proof of $1000$-digit Fibonacci Number}

\begin{document}
\maketitle

We want to invert the explicit formula for the Fibonacci numbers as to get an $\mrm{O}\prs{1}$ solution.
Since
$F_n$ is approximately $\frac{\phi^n}{\sqrt{5}}$, the index $n$ of $F_n$ will be approximated by $\log_\phi\prs{\sqrt{5} F_n}$.

\begin{theorem}
Let $F_n$ be the $n$\textsuperscript{th} Fibonacci number, for $n > 1$. Then if $F$ is a Fibonacci number, its index $n$ such that $F = F_n$ is given by $n\prs{F} = \brs{\log_\varphi\prs{\sqrt{5}F}}$, where $\brs{\cdot}$ is rounding to the nearest integer.
\end{theorem}

\begin{proof}
It is known that $F_n = \frac{\varphi^n - \psi^n}{\sqrt{5}}$, where $\varphi = \frac{1 + \sqrt{5}}{2}$ is the golden ratio and where $\psi = \frac{1 - \sqrt{5}}{2}$.
We get
\begin{align*}
\abs{F_n - \frac{\psi^n}{\sqrt{5}}} &= \abs{\frac{\psi^n}{\sqrt{5}}}
\\&= \frac{1}{\sqrt{5}} \abs{\psi^n}
\end{align*}
where $\abs{\psi^n} < \frac{1}{2}$ for $n > 1$ since
\begin{align*}
\abs{\psi} = \frac{\sqrt{5} - 1}{2} < \frac{3-1}{2} = 1 < \sqrt{2} \text{.}
\end{align*}
Hence
\begin{align*}
\abs{F_n - \frac{\psi^n}{\sqrt{5}}} \leq \frac{1}{2 \sqrt{5}} \text{.}
\end{align*}

Taking $N\prs{F} = \log_\varphi\prs{\sqrt{5} F}$, since $\varphi^x$ is convex in $x$ we get that
\begin{align*}
\abs{N\prs{F_n} - n} &\leq \abs{\varphi^{N\prs{F_n}} - \varphi^n}
\\&=
\abs{\varphi^{\log_{\varphi}\prs{\sqrt{5}F}} - \varphi^n}
\\&= \abs{\sqrt{5}F_n - \varphi^n}
\\&= \sqrt{5} \abs{F_n - \frac{\varphi^n}{\sqrt{5}}}
\\&\leq \frac{\sqrt{5}}{2 \sqrt{5}} = \frac{1}{2} \text{.}
\end{align*}
Hence
\begin{align*}
n = \brs{N\prs{F}} = \brs{\log_{\varphi}\prs{\sqrt{5}F}} \text{,}
\end{align*}
as required.
\end{proof}

\begin{corollary}
The minimal $n \in \mbb{N}$ such that $F_n$ has at least $k$ digits is one of the following
\begin{align*}
\brs{\log_{\varphi}\prs{\sqrt{5}} + \prs{k-1}\log_{\varphi}\prs{10}} \text{,} \\
\brs{\log_{\varphi}\prs{\sqrt{5}} + \prs{k-1}\log_{\varphi}\prs{10}} + 1 \text{.}
\end{align*}
\end{corollary}

\begin{proof}
We need to find $n$ such that $F_n \geq 10^{k-1}$ and $F_{n-1} \leq 10^{k-1}$. We have
\begin{align*}
n = \brs{\log_\varphi\prs{\sqrt{5} F_n}} \geq \brs{\log_\varphi\prs{\sqrt{5} \cdot 10^{k-1}}}
\end{align*}
and
\begin{align*}
n-1 = \brs{\log_\varphi\prs{\sqrt{5}F_{n-1}}} \leq \brs{\log_\varphi\prs{\sqrt{5} \cdot 10^{k-1}}} \text{.}
\end{align*}
Therefore, $n$ is either $\brs{\log_\varphi\prs{\sqrt{5} \cdot 10^{k-1}}}$ or $\brs{\log_\varphi\prs{\sqrt{5} \cdot 10^{k-1}}} + 1$.
Using basic properties of the logarithm, we get the result.
\end{proof}

\printbibliography

\end{document}