\documentclass[10pt]{article}

%---HYPERLINKS---

\RequirePackage{hyperref}
\hypersetup{
    colorlinks=true, %set true if you want colored links
    linktoc=all     %set to all if you want both sections and subsections linked
}

%---MATH---

\usepackage{math}

%---BIBLIOGRAPHY---

\usepackage[backend=biber, style=alphabetic]{biblatex}
\addbibresource{bibliography.bib}

%---TITLE---

\author{Alan Sorani}
\title{Proof of Correctedness for Highly Divisible Triangular Numbers}

\begin{document}
\maketitle

\begin{lemma}
Let
$m \in \mbb{N}_+$
and let
\begin{align*}
m = \prod_{i \in \brs{k}} p_i^{\alpha_i}
\end{align*}
be its prime factorization.
Then
\begin{align*}
\omega\prs{m} = \prod_{i \in \brs{k}} \prs{\alpha_i + 1} \text{,}
\end{align*}
where $\omega\prs{m}$ denotes the number of divisors of $m$.
\end{lemma}

\begin{proof}
The divisors of $m$ are the numbers of the form
$\prod_{i \in \brs{k}} p_i^{\beta_i}$
for $\beta_i \in \set{0, \ldots, \alpha_i}$, and since one can choose each $\beta_i$ separately and each choice of $\prs{\beta_i}_{i \in \brs{k}}$ gives a different divisor, we get the result.
\end{proof}

\begin{corollary}
For coprime $m,n \in \mbb{N}_+$ we have
\begin{align*}
\omega\prs{mn} = \omega\prs{m} \omega\prs{n} \text{.}
\end{align*}
\end{corollary}

\begin{corollary}
Let $T_n$ denote the $n$\textsuperscript{th} triangular number, i.e.
\begin{align*}
T_n = \sum_{k \in \brs{n}} k = \frac{n\prs{n+1}}{2} \text{.}
\end{align*}
We have
\begin{align*}
\omega\prs{T_n} =
\begin{cases}
\omega\prs{\frac{n}{2}} \omega\prs{n+1} & n \in 2\mbb{N}_+ \\
\omega\prs{n} \omega\prs{\frac{n+1}{2}} & n \notin 2 \mbb{N}_+
\end{cases}
\end{align*}
\end{corollary}

\printbibliography

\end{document}